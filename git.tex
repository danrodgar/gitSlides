% $Header: /cvsroot/latex-beamer/latex-beamer/solutions/generic-talks/generic-ornate-15min-45min.en.tex,v 1.5 2007/01/28 20:48:23 tantau Exp $

\documentclass{beamer}
%\documentclass[mathsherif]{beamer}

% This file is a solution template for:

% - Giving a talk on some subject.
% - The talk is between 15min and 45min long.
% - Style is ornate.



% Copyright 2004 by Till Tantau <tantau@users.sourceforge.net>.
%
% In principle, this file can be redistributed and/or modified under
% the terms of the GNU Public License, version 2.
%
% However, this file is supposed to be a template to be modified
% for your own needs. For this reason, if you use this file as a
% template and not specifically distribute it as part of a another
% package/program, I grant the extra permission to freely copy and
% modify this file as you see fit and even to delete this copyright
% notice.


\mode<presentation>
{
\usecolortheme[RGB={89,165,140}]{structure}

  \usetheme{Warsaw}
\setbeamercolor*{palette quaternary}{fg=white,bg=structure!40!black}


% o Singapore
% \setbeamercolor{normal text}{bg=blue!10} % para azul, la oscuridad del color se regula cambiando (!20)
% \beamertemplateshadingbackground{yellow!50}{magenta!50} % degradado de amarillo a magenta
  % or ...
% \setbeamertemplate{navigation symbols}{} quitar l\'{\i}nea de s\'{\i}mbolos esquina inferior derecha -in\'{u}tiles
  \setbeamercovered{transparent}
  % or whatever (possibly just delete it)
}


% \usepackage[spanish]{babel}
% or whatever

\usepackage[utf8]{inputenc}
% or whatever

\usepackage{times}
\usepackage[T1]{fontenc}
\usepackage{lmodern}

%\usepackage{lucidaso}
%\usepackage[small]{eulervm}

% Paquetes de David
\usepackage{verbatim}
\usepackage{listings}
\usepackage{color}
 
\definecolor{dkgreen}{rgb}{0,0.6,0}
\definecolor{gray}{rgb}{0.5,0.5,0.5}
\definecolor{mauve}{rgb}{0.58,0,0.82}
  
\lstset{ %
  language=C,                % the language of the code
  basicstyle=\footnotesize,           % the size of the fonts that are used for the code
  %numbers=left,                   % where to put the line-numbers
  numberstyle=\tiny\color{gray},  % the style that is used for the line-numbers
  numbersep=5pt,                  % how far the line-numbers are from the code
%  backgroundcolor=\color{white},      % choose the background color. You must add \usepackage{color}
  showspaces=false,               % show spaces adding particular underscores
  showstringspaces=false,         % underline spaces within strings
  showtabs=false,                 % show tabs within strings adding particular underscores
  %frame=single,                   % adds a frame around the code
  rulecolor=\color{black},        % if not set, the frame-color may be changed on line-breaks within not-black text (e.g. commens (green here))
  tabsize=2,                      % sets default tabsize to 2 spaces
  captionpos=b,                   % sets the caption-position to bottom
  breaklines=true,                % sets automatic line breaking
  breakatwhitespace=false,        % sets if automatic breaks should only happen at whitespace
  %title=\lstname,                   % show the filename of files included with \lstinputlisting;
  keywordstyle=\color{blue},          % keyword style
  commentstyle=\color{dkgreen},       % comment style
  stringstyle=\color{mauve},         % string literal style
  escapeinside={\%*}{*)},            % if you want to add a comment within your code
  morekeywords={*,...}               % if you want to add more keywords to the set
}


% Or whatever. Note that the encoding and the font should match. If T1
% does not look nice, try deleting the line with the fontenc.

% para Singapore
%\setbeamertemplate{footline}{%
%\leavevmode%
%\hbox{%
%\begin{beamercolorbox}[wd=.333333\paperwidth,ht=2.25ex,dp=1ex,center]{author in head/foot}%
%\usebeamerfont{author in head/foot}\insertshortauthor
%\end{beamercolorbox}%
%\begin{beamercolorbox}[wd=.333333\paperwidth,ht=2.25ex,dp=1ex,center]{title in head/foot}%
%\usebeamerfont{title in head/foot}\insertshorttitle
%\end{beamercolorbox}%
%\begin{beamercolorbox}[wd=.333333\paperwidth,ht=2.25ex,dp=1ex,right]{date in head/foot}%
%\usebeamerfont{date in head/foot}\insertshortdate{}\hspace*{2em}
%\insertframenumber{} / \inserttotalframenumber\hspace*{2ex}
%\end{beamercolorbox}}%
%\vskip0pt%
%}
%
% para Warsaw
\newcommand*\oldmacro{}%
\let\oldmacro\insertshorttitle%
\renewcommand*\insertshorttitle{%
  \oldmacro\hfill%
  \insertframenumber\,/\,\inserttotalframenumber}

%\renewcommand*{\appendixname}{Referencias}


\title[Git] % (optional, use only with long paper titles)
{Short Introduction to Git}

%\subtitle
%{Presentation Subtitle} % (optional)

\author[D Rodriguez] % (optional, use only with lots of authors)dvanced Robot Control
{Daniel Rodriguez}
% - Use the \inst{?} command only if the authors have different
%   affiliation.

\institute[University of Alcala] % (optional, but mostly needed)
{
  %\inst{1}%
  \textcolor{structure} %{\emph{\textbf{Computer Science Department} }}\\
  University of Alcala

% - Use the \inst command only if there are several affiliations.
% - Keep it simple, no one is interested in your street address.

%\date[Short Occasion] % (optional)
%{Date / Occasion}

\vspace*{0.5cm}
\includegraphics[height=0.8cm]{comun/uah}
}
\date{}



%logos s\'{o}lo en title
%\titlegraphic{
  %\includegraphics[scale=0.45]{comun/dpto}
  %\hfill
 % \includegraphics[scale=0.35]{gso1}
 % \hfill
 % \includegraphics[scale=0.20]{comun/gso1}
%}

% logos tal cual: salen en todos los frames...
%\pgfdeclareimage[height=0.4cm]{left-logo}{gso1}
%\pgfdeclareimage[height=0.4cm]{right-logo}{gso1}
%\logo{\pgfuseimage{right-logo}}


%\setbeamertemplate{sidebar left}
%{
%\logo{\pgfuseimage{left-logo}}
%\vfill%
%\rlap{\hskip0.1cm\insertlogo}%
%\vskip15pt%
%}

\subject{Talks}
% This is only inserted into the PDF information catalog. Can be left
% out.



% If you have a file called "university-logo-filename.xxx", where xxx
% is a graphic format that can be processed by latex or pdflatex,
% resp., then you can add a logo as follows:


% watermark

\usebackgroundtemplate{\includegraphics[width=\paperwidth]{comun/watermark}}


%% Remove %% from \AtBeginSubsection to generate links in the subsections
%%\AtBeginSubsection[]
%% {
%%     \begin{frame}{\'{I}ndice}
%  \small
%  \tableofcontents[currentsection,hideothersubsections]
%  \normalsize
% \end{frame}

%%    \small
%%    \tableofcontents[currentsection,currentsubsection]
     % \tableofcontents[pausesections]
%%   \end{frame}
%%}


%%\AtBeginSection[]
%% {
%%     \begin{frame}{\'{I}ndice}
%  \small
%  \tableofcontents[currentsection,hideothersubsections]
%  \normalsize
% \end{frame}

%%    \small
%%    \tableofcontents[currentsection]
     % \tableofcontents[pausesections]
%%   \end{frame}
%%}

% If you wish to uncover everything in a step-wise fashion, uncomment
% the following command:

%\beamerdefaultoverlayspecification{<+->}


\begin{document}

\begin{frame}
  \titlepage
\end{frame}

\begin{frame}{Table of Contents} %[shrink,plain]
 \frametitle{Table of Contents}
 \tableofcontents

 % no me vale: deja descolgado el cap\'{\i}tulo 4
 % \frame[allowframebreaks]%
 %    {\frametitle{\'{I}ndice}\tableofcontents[part=4]}
  % You might wish to add the option [pausesections]
\end{frame}


% Since this a solution template for a generic talk, very little can
% be said about how it should be structured. However, the talk length
% of between 15min and 45min and the theme suggest that you stick to
% the following rules:

% - Exactly two or three sections (other than the summary).
% - At *most* three subsections per section.
% - Talk about 30s to 2min per frame. So there should be between about
%   15 and 30 frames, all told.

\section{Software Configuration Management}

%%%%%%%%%%%%%%%%%%%%%%%%%%%%%%%%%%%%%%%%%%%%%%%%%%%%%%%%%%%%%%%%%%%%%%

\subsection{Version Control}


%%%%%%%%%%%%%%%%%%%%%%%%%%%%%%%%%%%%%%%%%%%%%%%%%%%%%%%%%%%%%%%%%%%%%%

\begin{frame}{Version Control}

Version control systems keep track of changes to source code.
Allows multiple people to edit a project in a predictable manner.

\end{frame}


%%%%%%%%%%%%%%%%%%%%%%%%%%%%%%%%%%%%%%%%%%%%%%%%%%%%%%%%%%%%%%%%%%%%%%

\begin{frame}[plain]{Software configuration Management}


Software configuration management is the task of tracking and controlling changes in the software, part of the larger cross-disciplinary field of configuration management.\\
(\begin{scriptsize}\texttt{https://en.wikipedia.org/wiki/Software\_configuration\_management}                                                                                         \end{scriptsize})

Main open source software configuration management systems
\begin{itemize}
 \item 1982 RCS
 \item 1990 CVS
 \item 2000 Subversion
 \item 2005 Git/Mercurial
\end{itemize}

There are many proprietary ones. 

All software should be under a configuration management system. If it isn't, that software doesn't exist!

\end{frame}

%%%%%%%%%%%%%%%%%%%%%%%%%%%%%%%%%%%%%%%%%%%%%%%%%%%%%%%%%%%%%%%%%%%%%%
\section{Git}

%%%%%%%%%%%%%%%%%%%%%%%%%%%%%%%%%%%%%%%%%%%%%%%%%%%%%%%%%%%%%%%%%%%%%
\begin{frame}{What is git?}

Git is an open source distributed version control system, created by Linus Torvald.

\texttt{https://git-scm.com/}

It is easier to star with free hosting sites instead of maintaining your own server.

\begin{itemize}
 \item Github: public repositories (as many as you want)
 \item Bitbucket: allow us to keep private repositories limiting the number of collaborators.
\end{itemize}

It is typically used as central repository:
\begin{itemize}
 \item from which everyone pulls other people’s changes
 \item to which everyone pushes changes they have made
\end{itemize}


\end{frame}

%%%%%%%%%%%%%%%%%%%%%%%%%%%%%%%%%%%%%%%%%%%%%%%%%%%%%%%%%%%%%%%%%%%%%%
\begin{frame}[fragile]{Init}

Initialization:

\begin{verbatim}
% mkdir /path/to/your/project
% cd /path/to/your/project
% git init
% git remote add origin https://<where>/<path>/<project.git>
\end{verbatim} 

% Then we can start tracking files. To do so, we need to add commit, and push the file(s) that we want to track. 
% 
% \begin{verbatim}
% echo "The participants are..." >> contributors.txt
% git add contributors.txt
% git commit -m 'Initial commit with contributors'
% git push -u origin master
% \end{verbatim} 

\end{frame}

%%%%%%%%%%%%%%%%%%%%%%%%%%%%%%%%%%%%%%%%%%%%%%%%%%%%%%%%%%%%%%%%%%%%%%
\begin{frame}{Cloning}

To work with someone else’s repository, we first need to clone it to get a
local copy.

\texttt{git clone <repo>}

Note: once cloned, you can edit the repository as much as you
want. No changes make their way back to the ‘central’ repository
until you explicitly do so.

\end{frame}

%%%%%%%%%%%%%%%%%%%%%%%%%%%%%%%%%%%%%%%%%%%%%%%%%%%%%%%%%%%%%%%%%%%%%%
\begin{frame}{diff}

\texttt{diff -u <old file> <new file>} 

shows you what changes you would need to apply to old file to change it into
new file.

Lines beginning with:
\begin{itemize}
 \item --- or +++ tell you the old / new filenames
 \item @@ tells you where within the file you're looking \\
       (i.e. a space) are lines that are unchanged
 \item  - is a deleted line
 \item + is a newly added line
\end{itemize}

\end{frame}


%%%%%%%%%%%%%%%%%%%%%%%%%%%%%%%%%%%%%%%%%%%%%%%%%%%%%%%%%%%%%%%%%%%%%%
\begin{frame}{Pulling}

To integrate all changes other people have made since you
cloned/pulled:

\texttt{git pull} .

\begin{itemize}
 \item If you have made local changes you have to git stash before
pulling, then git stash pop afterwards
\item You can see which files you’ve modified with git status
\item You can permanently remove your local changes by: \texttt{git
checkout <file>}
\end{itemize}

\end{frame}

%%%%%%%%%%%%%%%%%%%%%%%%%%%%%%%%%%%%%%%%%%%%%%%%%%%%%%%%%%%%%%%%%%%%%%
\begin{frame}{Pushing}

\texttt{git add <file>} makes git track the file <file> 

\texttt{git commit .} (notice the ‘.’) records all changes into a commit

\texttt{git push} pushes all new commits to the central repository
\end{frame}


%%%%%%%%%%%%%%%%%%%%%%%%%%%%%%%%%%%%%%%%%%%%%%%%%%%%%%%%%%%%%%%%%%%%%%
\begin{frame}{Merge and conflicts}

If two people both modify the same file, the first to push \emph{wins}.
The second person will have to pull and merge before pushing.

\begin{itemize}
 \item Changes in different parts of a file are automatically merged
 \item Changes in the same part of a file cause conflicts (between <<<
=== >>> ) and require the user to manually resolve them. Can
select either HEAD (your changes) or remote, or a mix of the two
\item Two merging cases: have / haven't committed
\end{itemize}

\end{frame}

%%%%%%%%%%%%%%%%%%%%%%%%%%%%%%%%%%%%%%%%%%%%%%%%%%%%%%%%%%%%%%%%%%%%%%
\begin{frame}{Commits}	

\begin{itemize}
 \item Merge commits record where parallel development unified
 \item How does git keep track of things when parallel development
happens?
\item Every commit has an ID (it's hash), which is a 40 character SHA-1
hash based on the commit's content. Not guaranteed to be
unique; but it probably is
\end{itemize}

\end{frame}

%%%%%%%%%%%%%%%%%%%%%%%%%%%%%%%%%%%%%%%%%%%%%%%%%%%%%%%%%%%%%%%%%%%%%%
\begin{frame}{Branches}

Branches are used extensively (e.g. some like feature branches).

\begin{itemize}
 \item A repository (local and remote) can have explicit branches
 \item The default branch is called master
 \item \texttt{git branch <name>} creates branches
 \item \texttt{git checkout <branch name>} switch branches
 \item To merge branch X into Y, checkout Y and run git merge X
(i.e. you say “I want to merge another branch into me”)
 \item 
\end{itemize}


\end{frame}


%%%%%%%%%%%%%%%%%%%%%%%%%%%%%%%%%%%%%%%%%%%%%%%%%%%%%%%%%%%%%%%%%%%%%%
\begin{frame}{Good practices}

One of the best techniques for improving code quality is code
review: changes are checked by someone other than their
author before being merged into master. 

This workflow is naturally captured by pull requests.

Learn on the job: the best way to learn it is by using it. However:
\begin{itemize}
  \item Best practice: regularly push and pull (at least daily, in general).
  \item Don't  push half-finished changes or pull if you're in the middle of something
\end{itemize}



\end{frame}



% %%%%%%%%%%%%%%%%%%%%%%%%%%%%%%%%%%%%%%%%%%%%%%%%%%%%%%%%%%%%%%%%%%%%%%
% \begin{frame}{}
% 
% \end{frame}
% 


%%%%%%%%%%%%%%%%%%%%%%%%%%%%%%%%%%%%%%%%%%%%%%%%%%%%%%%%%%%%%%%%%%%%%%

\end{document}





